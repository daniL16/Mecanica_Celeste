\documentclass[10pt,a4paper]{article}
\usepackage[latin1]{inputenc}
\usepackage{amsmath}
\usepackage{amsfonts}
\usepackage{amssymb}
\usepackage{graphicx}

\begin{document}
	Definici�n. Se define la elipse como el conjunto de puntos cuya suma de distancia a dos puntos dados(llamados focos) es constante.\\
	
	\bigskip Consideramos como focos el punto A y el origen. Tomamos un punto x cualquiera de la elipse y por la definici�n tenemos que $|x|+|A-x|=c$ con $c$ una constante positiva.\\
	\medskip

	Operando sobre esa igualdad tenemos que:\\

	\medskip
	$|A-x|= c - |x| \implies {|A-x|}^2 = {(c-|x|)}^2 \\
	\implies  |A|^2+|x|^2-2<A,x> = c^2+|x|^2-2c|x| \\
	\implies |A|^2-2<A,x>= c^2 -2c|x| 
	\\ \implies |x|+<-\frac{1}{c}A,x> = \frac{c^2-|A|^2}{2c}$\\
	
	\bigskip
	
	Tomando $e = -\frac{1}{c}A $ y $k = \frac{c^2-|A|^2}{2c}$ tenemos la ecuaci�n de una c�nica.\\
	Finalmente, usando la desigualdad triangular: \\
	\medskip
	$|A| < |x| + |A-x| = c $ \\
	Luego $|e| < 1 $ y $k > 0$ 
\end{document}